\documentclass[12pt]{article}
\usepackage[utf8]{inputenc}
\usepackage[spanish]{babel}
\usepackage[top=1in,bottom=1in,left=1in,right=1in]{geometry}

%paquetes de matematica
\usepackage{amsmath}
\usepackage{amssymb}
\usepackage[mathscr]{euscript}


%comandos de secciones 
\newcommand{\teorema}{\section{Teorema}}
\newcommand{\definicion}{\section{Definición}}
\newcommand{\ejercicios}{\section{Ejercicios}}
\newcommand{\solucion}{\section{Solución}}

%comandos varios utiles (agregar es posible)
\newcommand{\Rn}[1]{\mathbb{R}^{#1}}
\newcommand{\vect}[1]{\textbf{#1}}
\newcommand{\vecti}[2]{\textbf{#1}_{#2}}
\newcommand{\bola}[2]{\mathcal{B}_{#1}(#2)}
\newcommand{\parcial}[2]{\frac{\partial #1}{\partial #2}}

\title{Clase \# 7 de Análisis 3}
\author{Equipo clases a \LaTeX}

\begin{document}
	
	\maketitle
	\tableofcontents
	
	\teorema 
	\underline{Condición suficiente de diferenciabilidad} \\
	
	Sea  $f \, : \, S \subset \Rn{n} \mapsto \Rn{}$, $S$ abierto, $\vec{x} \in S$. $ D_1 f, \cdots, D_n f $ existen y son continuas en $B(\vec{x}, r) \subset S$. Entonces $f$ es diferenciable en $\vec{x}$

	\teorema
	\underline{Regla de la cadena} \\
	
	Consideremos $f \, : \, S \subset \Rn{n} \mapsto \Rn{}$, $S$ abierto, $\vec{r} \, : \, I \subset \Rn{} \mapsto S$, y $g(t) = f(\vec{r}(t))$, $t \in I$. Sea $t \in I$ donde $\vec{r} \, ^{\prime} (t)$ existe y supongamos que $f$ es diferenciable en $\vec{r}(t)$, entonces existe $g^{\prime} (t)$ y tenemos que 
	
	$$ g ^{\prime} (t) = \nabla  f(\vec{x}) \cdot \vec{r}\,^{\prime} (t)$$
	
	donde $\vec{x} = \vec{r} (t)$.
	
    \pagebreak
    

	\ejercicios
	
	\begin{enumerate}
		\item \textit{Halle  el vector gradiente si}
		
		\begin{enumerate}
		    \item $f(x,y) \, = \, x^{2} \, + \, y^{2} \sin(xy).$
		    \item $f(x,y,z) \, = \, x^{2} \, - \, y^{2} \, +2z^{2}.$
		\end{enumerate}
		
		\item \textit{Calcule la derivada direccional de $f(x,y,z) \, = \, x^{2} \, + \, 2y^{2} \, + \, 3z^{2}$ en $(1,1,0)$ en la dirección de $\vec{e}_{1} - \vec{e}_{2} + 2\vec{e}_{3}$.}
		
		\item \textit{Hallar los puntos (x,y) y las direcciones para las que la derivada direccional de $f(x,y) \, = \, 3x^{2} \, + \, y^{2}$ tiene valor máximo, si (x,y) pertenece a la circunferencia $x^{2} \, + \, y^{2} \, = \, 1$.}
		
		\item \textit{Supóngase que f es diferenciable en cada punto de $B(\vec{x}, r)$. Demuestre:}
		
		\begin{enumerate}
		    \item Si $\nabla f(\vec{y}) = \vec{0}$ para todo $\vec{y} \in B(\vec{x}, r)$ entonces f es constante en $B(\vec{x}, r)$.
		    \item Si $f(\vec{y}) \leqslant f(\vec{x})$ para todo $\vec{y} in B(\vec{x}, r)$ entonces $\nabla f(\vec{x}) = \vec{0}.$
		\end{enumerate}
		
		\item \textit{Hallar la derivada direccional de $f(x,y) \, = \, x^{2} \, - \, x \, + \, 2$ a lo largo de $y \, = \, x^{2} \, - \, x \, + \, 2$ en el punto (1,2). Use regla de la cadena.}
		
		\item \textit{Sea f un campo escalar no constante diferenciable en todo el plano y c una constante. Supongamos que la ecuación $f(x,y) \, = \, c$ describe una curva $\mathscr{C}$ que tiene tangente en cada uno de sus puntos. Demuestre que f tiene las siguientes propiedades en cada punto de $\mathscr{C}$}
		
		\begin{enumerate}
		    \item $\nabla f$ es un vector normal a $\mathscr{C}$.
		    \item La derivada direccional de f a lo largo de $\mathscr{C}$ es cero.
		    \item La derivada direccional de f tiene su valor máximo en la dirección del vector normal a $\mathscr{C}$.
		\end{enumerate}
		
		\item \textit{Sea $f \, : \, S \subset \Rn{3} \mapsto \Rn{}$, $S$ es abierto, f es diferenciable en $S$. Sea c una constante y consideremos la superficie de nivel $\mathscr{H} \, = \, \{\vec{y} \in S \, ; \, f(\vec{y}) \, = \, c \}$. Sea $\vec{a} in \mathscr{H}$. Demuestre que la ecuación del plano tangente a la superficie $\mathscr{H}$ satisface la ecuación}
		
		$$ \nabla f (\vec{a}) \cdot (\vec{x} - \vec{a}) \, = \, 0 $$
		
		\item \textit{Sea $f(x,y) \, = \, \sqrt{| xy |}$. Compruebe que $\dfrac{\partial f}{\partial x} \, = \, \dfrac{\partial f}{\partial y} \, = \, 0$ en (0,0) ¿Tiene la superficie $z \, = \, f(x,y)$ plano tangente en (0,0)?}
		
	\end{enumerate}
	
	\pagebreak 
	
	\solucion
	
\end{document}